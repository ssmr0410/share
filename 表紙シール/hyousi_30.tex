\documentstyle[my_12,epsbox]{my_jreport}
\pagestyle{myheadings}
\headsep 70pt
\topmargin -15mm
\oddsidemargin 1cm
\textwidth 440pt
\textheight 55pc
\setlength{\footskip}{70pt}
\setlength{\baselineskip}{30pt}
\setlength{\parskip}{6pt}

\renewcommand{\textfraction}{0}

\newenvironment{indention}[1]{\par
\addtolength{\leftskip}{#1}
\begingroup}{\endgroup\par}

\newcommand{\namelistlabel}[1]{\mbox{#1}\hfil}
\newenvironment{namelist}[1]{%
\begin{list}{}
{\letlmakelabel\namelistlable
\settowidth{\labelwidth}{#1}
\setlength{\leftmargin}{1.1\labelwidth}}
}{%
\end{list}}

\def\theenumi{\roman{enumi}}
\newcommand{\q}{\hspace*{1em}}


\begin{document}
\begin{titlepage}
\begin{center}
\noindent

{\Large 平成30年度 卒業論文}

\vspace{0.15cm}

{\LARGE 密でないグラフに対して}

{\LARGE 最大k-plexを発見する}

{\LARGE 準指数時間アルゴリズム}

{\Large 
\vspace{2cm}
\begin{tabular}{rl}
指導教員 泉 泰介 准教授 \\
名古屋工業大学 情報工学科 \\
平成27年度入学 27115067 番 
\end{tabular}}

{\LARGE 佐藤 僚祐}

\vspace*{2cm}
\noindent

{\Large 平成30年度 卒業論文}

\vspace{0.15cm}

{\LARGE スパイダグラフ上の最短トークン遷移に対する}

{\LARGE 線形時間アルゴリズム}


{\Large 
\vspace{2cm}
\begin{tabular}{rl}
指導教員 泉 泰介 准教授\\
名古屋工業大学 情報工学科\\
平成27年度入学 27115138 番
\end{tabular}}

{\LARGE 鉾館 歩}

\vspace*{2cm}
\noindent
\newpage


{\Large 平成30年度 卒業論文}

\vspace{0.15cm}

{\LARGE 本埋め込みを利用した平面的グラフの簡潔表現}

{\Large 
\vspace{2cm}
\begin{tabular}{rl}
指導教員 泉 泰介 准教授\\
名古屋工業大学 情報工学科\\
平成27年度入学 27115169 番
\end{tabular}}

{\LARGE 細川 秀樹}

\vspace*{2cm}
\noindent

{\Large 平成30年度 卒業論文}

\vspace{0.15cm}

{\LARGE ランキング距離計算の2者間通信複雑性}

{\Large 
\vspace{2cm}
\begin{tabular}{rl}
指導教員 泉 泰介 准教授\\
名古屋工業大学 情報工学科\\
平成27年度入学 27115139 番
\end{tabular}}

{\LARGE 前田 隼}

\vspace*{2cm}
\noindent
\newpage


{\Large 平成30年度 卒業論文}

\vspace{0.15cm}

{\LARGE 平衡分離集合を近似する分散アルゴリズム}

{\Large 
\vspace{2cm}
\begin{tabular}{rl}
指導教員 泉 泰介 准教授\\
名古屋工業大学 情報工学科\\
平成26年度入学 26115142 番
\end{tabular}}

{\LARGE 水谷 龍誠}

\vspace*{2cm}
\noindent


\end{center}
\end{titlepage}
\end{document}
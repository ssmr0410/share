\documentclass{jarticle}
%\bibliographystyle{junsrt}

\usepackage[dvipdfmx]{graphicx}
\usepackage{amsmath}			% math系
\usepackage{amssymb}			% math系
\usepackage{float}				% 図表の挿入箇所を固定する[H]指定
\usepackage{cite}				% 参考文献
\usepackage{url}				% 参考文献中のURL表記
\usepackage{algorithm}			% アルゴリズム環境
\usepackage{algpseudocode}      %アルゴリズム環境
\usepackage{listings}  			%アルゴリズム環境

\let\thealgorithm\relax %タイトル番号非表示

\usepackage{comment}			% コメントアウト環境
\usepackage{here}               % 定理環境
\usepackage{citesort}

% 先生コメント用
\usepackage[normalem]{ulem}
\usepackage{color}
\newcommand{\Izumi}[1]{\textcolor{blue}{#1}}
\newcommand{\Izurep}[2]{\textcolor{red}{\sout{#1}}{\Izumi{#2}}}

\begin{document}


\section*{アルゴリズム2}

\begin{algorithm}[H]
    \caption{Find Innermost s-sided Cut of Minimum Size} 
    \textbf{初期設定:} \par
    $g(s),g(t)\in \mathbb{N}$ \par
    点素s-tパス$f_1,...,f_k$ \par
    $\ell_i:f_i$の長さ($(s)$は0番目) \par
    $v_{ij},0 \leq j \leq \ell_i$を$f_i$上の$j$番目の頂点とする \par
    $w_i=v_{i1}(1\leq i \leq k)$ \par
    valid := \textbf{false} \par 
    {\setlength{\baselineskip}{10pt}
    \begin{algorithmic}[1]
      \For {$i=1$ to $k$}
        \State {$c:=0$} \Comment {$f_i$の始まり}
        \State {$d:=\ell_i$} \Comment {$f_i$の終わり}
        \While {$d \neq c+1$}
            \State {$e:= \lceil \frac{c+d}{2} \rceil$} \Comment {$f_i$二分探索}
            \State {$W:=w_1,...,w_{i-1},v_{ie},w_{i+1},...,w_k$} \Comment {$i$番目を動かす}
            \If {$W^+\neq (t,t,...,t)$and$|V_s(W^+)+g(s)\leq |V_r(W^+)\cup V_t(W^+)|+g(t)$} 
            \State {$c:=e$} 
            \State {valid $:=$ \textbf{true}}
            \Else
            \State {$d:=e$}
            \EndIf 
        \EndWhile \Comment {$O(\log n)$回反復}
      \State {$w_i:=v_{ic}$}
      \EndFor
      \If {vaild}
      \State {\textbf{return}} $(w_1,...,w_k)$
      \Else
      \State {\textbf{return}} $(s,s,...,s)$
      \EndIf
    \end{algorithmic}
    }
\end{algorithm}

\section*{アルゴリズム3}

\begin{algorithm}[H]
  \caption{Find Innermost t-sided Cut of Minimum Size} 
  \textbf{初期設定:} \par
  $g(s),g(t)\in \mathbb{N}$ \par
  点素s-tパス$f_1,...,f_k$ \par
  $\ell_i:f_i$の長さ($(s)$は0番目) \par
  $v_{ij},0 \leq j \leq \ell_i$を$f_i$上の$j$番目の頂点とする \par
  $w_i=v_{i1}(1\leq i \leq k)$ \par
  valid := \textbf{false} \par 
  {\setlength{\baselineskip}{10pt}
  \begin{algorithmic}[1]
    \For {$i=1$ to $k$}
      \State {$c:=0$} \Comment {$f_i$の始まり}
      \State {$d:=\ell_i$} \Comment {$f_i$の終わり}
      \While {$d \neq c+1$}
          \State {$e:= \lfloor \frac{c+d}{2} \rfloor$} \Comment {$f_i$二分探索}
          \State {$W:=w_1,...,w_{i-1},v_{ie},w_{i+1},...,w_k$} \Comment {$i$番目を動かす}
          \If {$W^-\neq (t,t,...,t)$and$|V_t(W^-)+g(t)\leq |V_r(W^-)\cup V_s(W^-)|+g(s)$} 
          \State {$d:=e$} 
          \State {vaild $:=$ \textbf{true}}
          \Else
          \State {$c:=e$}
          \EndIf 
          \EndWhile \Comment {$O(\log n)$回反復}
    \State {$w_i:=v_{id}$}
    \EndFor
    \If {vaild}
    \State {\textbf{return} $(w_1,...,w_k)$}   
    \Else
    \State {\textbf{return}} $(s,s,...,s)$
    \EndIf
  \end{algorithmic}
  }
\end{algorithm}

アルゴリズム2,3共に$O(km\log n)$時間

\section*{分散案}

\begin{lemma}[経路収集]
  $G$の有向パス$P = \{v_1,...,v_\ell\}$を考える。ここで、各ノード$v_i$はそのパスの先頭ノードと後尾ノードを知っている。$O(\log n)$SAラウンドにおいて、各ノード$v_i$はiの値と経路内のそのインデックスを知ることができる。さらに、各ノード$v_i$が整数$x_i$と共通の結合演算子$\oplus$を知っていれば、各ノード$v_i$に先頭集約$\bigoplus_{j\leq i}xj$と後尾集約$\bigoplus_{j\geq i}x_j$を学習させることができる。
\end{lemma}
パスの長さやインデックスは$O(\log n)$SAラウンドで計算可能.二分探索はパス上のノードだけが計算すればよさそう?


\end{document}
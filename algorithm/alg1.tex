\documentclass{jarticle}
%\bibliographystyle{junsrt}

\usepackage[dvipdfmx]{graphicx}
\usepackage{amsmath}			% math系
\usepackage{amssymb}			% math系
\usepackage{float}				% 図表の挿入箇所を固定する[H]指定
\usepackage{cite}				% 参考文献
\usepackage{url}				% 参考文献中のURL表記
\usepackage{algorithm}			% アルゴリズム環境
\usepackage{algpseudocode}      %アルゴリズム環境
\usepackage{listings}  			%アルゴリズム環境
\usepackage{comment}			% コメントアウト環境
\usepackage{here}               
\usepackage{color}
\usepackage{citesort}

% 定理環境
\usepackage{amsthm}
\theoremstyle{definition}
\newtheorem{theorem}{定理}
\newtheorem{lemma}{補題}
\newtheorem{definition}{定義}
\newtheorem*{definition*}{定義}
\newtheorem{fact}{事実}
\newtheorem{corollary}{系}

% 先生コメント用
\usepackage[normalem]{ulem}
\usepackage{color}
\newcommand{\Izumi}[1]{\textcolor{blue}{#1}}
\newcommand{\Izurep}[2]{\textcolor{red}{\sout{#1}}{\Izumi{#2}}}

\begin{document}
\section*{諸定義}
\begin{definition}
  ${f_1,...,f_k}$をs-t点素パスの集合とする。
  $U=(u_1,...u_k)$および$W=(w_1,...,w_k)$を$(f_1,...,f_k)$に関するスライスとし、すべての$1\leq i\leq k$について、$u_i$は$f_i$における$w_i$の前者または$u_i=w_i$とする。この時$U$は$W$より$S$に近いと言い、$U\preceq W$と書く。もし$1 \leq i \leq k$に対して、さらに$u_i \neq w_i$ならば、$U$は$W$より厳密に$s$に近いと言い、$U \prec W$と書く。同様に、$W$は$U$より厳密に$t$に近いと言う。便宜上、タプル$(s,s,...,s)$と$(t,t,...,t)$についても同様に上記を定義。したがって、例えば、$(s,s,...s)$はどのスライスよりもsに近いと言える。$"\preceq"$は一般的な順序の合計を定義するものではない。
\end{definition}

\begin{definition}
  $U$を任意のカットとする。$s$を含む$G [V\backslash U]$の連結成分の頂点セットとして$V_s(U)$を定義し、$t$を含む$G[V\backslash U]$の連結成分の頂点セットとして$V_t(U)$を定義し、$V_r$を$G[V \backslash U]$の残りの連結成分の頂点集合の和集合として定義する。すなわち、$s$も$t$も含まないものである(したがって、$V_r(U)$は空であり得る)。
\end{definition}

\begin{definition}
  $(f_1,...,f_k)$に関してUをスライスとする。
  $U\preceq X$であり、かつ$U\preceq X'\prec X$を満たすカット$X'$が存在しないようなカットを$X$とする。この時$U^+:=X$を定義する。\\
  上記のような$X$が存在しない場合は、$U^+:=(t,t,...,t)$とする。\\
  同様に、$Y\preceq U$かつ$U\preceq Y'\prec Y$を満たすカット$Y'$が存在しないようなカットを$Y$とする。この時、$U^-:=Y$を定義する。\\
  上記のような$Y$が存在しない場合は$U^-:=(s,s,...,s)$とする 
\end{definition}

\begin{definition}[Subgraph Aggregation]
  $G=(V,E)$をネットワークグラフとし、$\mathcal{P} =(P_1,...P_{|\mathcal{P}|})$をpartの集合とし、各$P_i$について$H_i$を$P_i$のノード上のGの連結部分グラフとする。必ずしもグラフ$G[P_i]$から誘導されるとは限らない。各部分グラフ$H_i$について、$V(H_i)$内のすべてのノードが部分グラフ$H_i$内の隣接ノードを認識し、それ以外は何も知らないと仮定する。すべてのノード$v\in \bigcup_iP_i$がO(log n)ビットの整数$x_v$を持ち、$\oplus$を長さO(log n)の整数に作用する結合関数とする。$P_i$内の各ノードは値$\bigoplus_{v\in P_i}x_v$、すなわち$P_i$内のすべての値$x_v$の集合$\oplus$を知りたいとする。このようなタスクをオペレーター$\oplus$におけるSubgraph Aggregationと呼ぶ。
\end{definition}

\section*{アルゴリズム1}

\begin{algorithm}[H]
        \caption{$U^+$の計算} 
        \textbf{初期設定:} \par
        点素s-tパス$f_1,...,f_k$ \par
        スライス$U=(u_1,...,u_k)$ \par
        集合$X=\{\},x\in X$ \par 
        $w_i := u_i$ \Comment {}\par
        {\setlength{\baselineskip}{10pt}
        \begin{algorithmic}[1]
          \For {$i=1$to$k$}
          \State {$w_i$の前にあるノードをすべて$X$に入れる} \Comment {$X$は$V_s(U^+)$の候補}
          \EndFor
          \State {$Y:=V\backslash(X\cup U),y\in Y$}   
          \While {$\{x,y\}\in E$がなくなるまで}
            \If {$y=t$}
            \State \textbf{return} $(t,t,...,t)$
            \ElsIf {$y$が$f_i$に属している}
            \State {$X$に$f_i$上で$y$より前にある頂点を加える}
            \State {$Y$から$X$に加えた頂点を削除}
            \State {$w_i=y$}
            \Else 
            \State {頂点$y$を$X$にを入れて$Y$から削除する}
            \EndIf
          \EndWhile
          \State {\textbf{return}} $(w_1,...,w_k)$
        \end{algorithmic}
        }
\end{algorithm}
$U^+$と$U^-$の計算時間は$O(m)$

\begin{comment}
\section*{分散案1}

\begin{enumerate}
  \item 各$u_i$は$w_i:=u_i$としてパスに沿って$s$方向に送信
  \item ノード$s$から$w_i$のインデックスをブロードキャスト
  \item 受信するノードがs-tパス上にある時、そのノードは$w_i$の値を更新してパスに沿って$s$方向にのみ送信
  \item 受信した$w_i$の値がパス上の自身のインデックスと等しいとき、$w_i$をグラフ全体にブロードキャスト($u_i$の時は除外)
  \item $t$が値を受信したとき、$w_i:=t$としてグラフ全体にブロードキャスト
\end{enumerate}
$O(kn)$ラウンドで実行可能(?)
\end{comment}

\newpage

\section*{分散案}
\begin{enumerate}
  \item 点素パス上の各頂点が隣接頂点にパス上のインデックスをブロードキャスト
  \item 各点素パスの頂点を含まない$G$の連結成分内(仮$C$)で,受け取ったパスのインデックスの最大値最小値を収集(2kSAラウンド)
  \item スライス$U$のインデックスを各点素パス上のノードが知る(kSAラウンド)
  \item パス上の各頂点がスライス$U$のインデックスを隣接頂点にブロードキャスト(kラウンド)
  \item 各$C$内の頂点が点素パスの情報を2つ以上知っているとき
  \begin{itemize}
    \item 各$C$内で受け取ったスライスのインデックスと大小比較
    \item どれか一つでもスライスのインデックスより小さい時,$C$を$X$に追加
    \item $X$内でパスの最大値を収集(kSAラウンド)
  \end{itemize}
  \item loop 一番長い点素パスの長さ$\ell$分だけ繰り返す可能性
  \item 収集したインデックスの最大値(仮$w_i$)の1つ前までのパスの頂点を$X$に追加
  \item 最小値が$x_i$より小さい$C$を$X$に追加
  \item 最大値収集,$w_i$を更新
\end{enumerate}
$\ell k$SAラウンド

\end{document}
\documentclass{thesis}
\bibliographystyle{unsrt}

\input{jdummy.def}

\usepackage[dvipdfmx]{graphicx}

\usepackage{amsmath}			% math系
\usepackage{amssymb}			% math系
%\usepackage{float}				% 図表の挿入箇所を固定する[H]指定
\usepackage{cite}				% 参考文献
%\usepackage{url}				% 参考文献中のURL表記
\usepackage{algorithm}			% アルゴリズム環境
\usepackage{algorithmic}		% アルゴリズム環境
\usepackage{comment}			% コメントアウト環境
\usepackage{bm}	%太字形式のベクトル

\headsep=1.4cm  %本文上にスペースを空けたい場合は 20mm にする

% 定理環境
\usepackage{amsthm}
\theoremstyle{definition}
\newtheorem{theorem}{定理}
\newtheorem{lemma}{補題}
\newtheorem{definition}{定義}
\newtheorem*{definition*}{定義}
\newtheorem{fact}{事実}
\newtheorem{corollary}{系}

% 先生コメント用
\usepackage[normalem]{ulem}
\usepackage{color}
\newcommand{\Izumi}[1]{\textcolor{blue}{#1}}
\newcommand{\Izurep}[2]{\textcolor{red}{\sout{#1}}{\Izumi{#2}}}

%%%%%%%% ここか本体 %%%%%%%%%%%%%%%%%%%%%%%%

\begin{document}
\baselineskip=22pt
\pagestyle{empty}

% タイトル
\gradyear{30}
\papertitle{タイトル}
\IDNumber{26115142}
\department{情報工学科}
\labo{泉研究室}
\enteryear{26}
\name{水谷 龍誠}
\maketitle

% 目次
\pagestyle{myheadings}	% ページ番号を右上につける
\pagenumbering{roman}	% ページ番号をローマ数字で
\tableofcontents

\newpage

% 本文
\pagenumbering{arabic}	% ページ番号をアラビア数字で

\chapter{はじめに}

\section{研究背景}

あるグラフ$G=(V,E)$が与えられたとき,グラフを非連結な二つの頂点集合に分割する小さな"セパレータ"の存在について考えられることがある.このセパレータの存在は,高速なグラフアルゴリズム設計において非常に重要である.
しかし,一般のグラフに対して最小サイズのセパレータを求める問題はNP困難である.\par
現在,一般のグラフに対して小さなセパレータを近似する集中型のアルゴリズムはいくつか知られているが,分散環境におけるアルゴリズムはまだあまり知られていない.この論文では,既存の近似アルゴリズムをベースとする分散セパレータ近似アルゴリズムを提示する.

\section{研究結果}

\section{関連研究}

\section{論文の構成}

\chapter{準備}
\section{分散モデル}
アルゴリズムは分散システムにおけるCONGESTモデルの下で動作する.$n$個の計算機ノード集合$V$と通信リンクの集合$E$である無向ネットワークグラフ$G=(V,E)$があるとする.CONGESTモデルにおいて計算機はラウンドに従って同期して動作を行う.各ラウンドにおいてノードは最大$O(\log n)$ビットのメッセージを各隣接ノードに送信,各隣接ノードからメッセージの受信,無制限の内部計算を行う事が出来る.\\
分散設定では,ノードは一意なID($O\log n$ビット)を持っており,隣接ノードの$ID$は知っているものとする.\par
グラフ$G=(V,E)$に対して,$D$はその直径を表すとする.頂点$s,t\in V$に対して,$s$と$t$を結ぶ経路をs-tパスと呼ぶ.特に,いくつかのs-tパスに対して,それぞれのパスが同じ頂点を共有しないとき,これらをs-t点素パスと呼ぶ.また,$s$と$t$の間にパスが存在しなくなるように取り除かれた頂点集合をカットと呼ぶ.


\section{Subgraph Aggregation}
GhaffariとHaeuplerのショートカットフレームワークは,制限されたグラ
フ族の分散アルゴリズムを設計する上で有益であることが証明されて
いる.さらに,このショートカットフレームワークの制約をさらに弱めて改良されたタスクがJasonによって示されている.
この論文では,このショートカットのフレームワークの内部動作には触れずにSubgraph Aggregationとして定義されているタスクを利用する.

\begin{definition}[Subgraph Aggregation]
    $G=(V,E)$をネットワークグラフとし,$\mathcal{P} =(P_1,...P_{|\mathcal{P}|})$をパートの集合,各$P_i$について$H_i$を$P_i$のノード上のGの連結部分グラフとする.必ずしもグラフ$G[P_i]$から誘導されるとは限らない.各部分グラフ$H_i$について、$V(H_i)$内のすべてのノードが部分グラフ$H_i$内の隣接ノードを認識し、それ以外は何も知らないと仮定する.すべてのノード$v\in \bigcup_iP_i$がO(log n)ビットの整数$x_v$を持ち,$\oplus$を長さO(log n)の整数に作用する結合関数とする.$P_i$内の各ノードは値$\bigoplus_{v\in P_i}x_v$、すなわち$P_i$内のすべての値$x_v$の集合$\oplus$を知りたいとする.このようなタスクをオペレーター$\oplus$におけるSubgraph Aggregationと呼ぶ.
\end{definition}

SAラウンドを,Subgraph Aggregation(SA)において各パートがその収集値$\bigoplus$を学習する一回の反復とする.また,SAは特別なグラフ構造に対して以下の定理が成り立つ.

\begin{theorem}
    結合演算子$\oplus$について、$Q_G$がグラフ$G$とその直径$D$に依存するパラメータである場合,$\tilde{O}(Q_G)$ラウンドでSubgraph Aggregation問題を解くことができる.
    \item 全てのグラフ$G:Q_G = O(\sqrt{n}+D)$
    \item 種数$g$のグラフ$G:Q_G=O(\sqrt{g+1}D)$
    \item 木幅$k$のグラフ$G:Q_G=\tilde{O}(kD)$
    \item $H$をマイナーとして含まないグラフ$G:Q_G=\tilde{O}(f(H)\cdot D^2)$,$f$は$H$にのみ依存する関数
\end{theorem}

\section{s-t点素パス}
グラフプロパティの一種である木幅と呼ばれる値$k$に対して以下の補題が示されている.

\begin{lemma}
    木幅が高々$k$のグラフ$G=(V,E)$と二つの頂点$s,t\subseteq V$を与えると,$k$点素s-tパスを見つけるか,サイズ$k$以下のs-tノードカットを$\tilde{O}(k^{O(1)}D)$ラウンドで出力することができる.前者の場合,各ノードは自身がパス上にあるかどうかを知っており、そうであれば、そのパス上のその前方と後方を知る.後者の場合、$k$点素パスが存在しないという事実と,カットに含まれるかどうかを各ノードが知っている.
\end{lemma}

 アルゴリズムの詳細については触れないが,点素パスの探索ステップ自体には木幅の値は関わっていない.木幅が影響するのはSAラウンドに関してのみである.上記の補題と定理1より一般のグラフに置き換えた以下の推論が立てられる.

\begin{corollary}
    補題1のグラフ$G$を一般のグラフに置き換えると,最大$\ell$本のs-tパスは$\tilde{O}(\ell^{O(1)}(\sqrt{n}+D))$ラウンドで見つけることができる.
\end{corollary}


\chapter{既存手法の説明}
test

\chapter{まとめと今後の課題}
test


\chapter{謝辞}
本研究の機会を与え,数々の御指導を賜りました泉泰介准教授に深く感謝致します.また,本研究を進めるにあたり多くの助言を頂き,様々な御協力を頂きました泉研究室の学生の皆様に深く感謝致します.

% 参考文献

\bibliography{newmain}

% 付録
%\include{appendix}

\end{document}
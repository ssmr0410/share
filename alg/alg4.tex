\documentclass{jarticle}
%\bibliographystyle{junsrt}

\usepackage[dvipdfmx]{graphicx}
\usepackage{amsmath}			% math系
\usepackage{amssymb}			% math系
\usepackage{float}				% 図表の挿入箇所を固定する[H]指定
\usepackage{cite}				% 参考文献
\usepackage{url}				% 参考文献中のURL表記
\usepackage{algorithm}			% アルゴリズム環境
\usepackage{algpseudocode}      %アルゴリズム環境
\usepackage{listings}  			%アルゴリズム環境

\let\thealgorithm\relax %タイトル番号非表示

\usepackage{comment}			% コメントアウト環境
\usepackage{here}              
\usepackage{color}
\usepackage{citesort}

% 定理環境
\usepackage{amsthm}
\theoremstyle{definition}
\newtheorem{theorem}{定理}
\newtheorem{lemma}{補題}
\newtheorem{definition}{定義}
\newtheorem*{definition*}{定義}
\newtheorem{fact}{事実}
\newtheorem{corollary}{系}

% 先生コメント用
\usepackage[normalem]{ulem}
\usepackage{color}
\newcommand{\Izumi}[1]{\textcolor{blue}{#1}}
\newcommand{\Izurep}[2]{\textcolor{red}{\sout{#1}}{\Izumi{#2}}}

\begin{document}

\section*{アルゴリズム4}

\begin{algorithm}[H]
    \caption{Find Innermost Cut of Bounded Size}
    \textbf{初期設定:} \par
    正の整数$K$と二つの頂点$s,t\in V$ \par
    $g(s):= g(t) := 1$ \par
    $k:=0$ \par
    $H:=G$かつ$S:=T:=\{\}$ \par
    {\setlength{\baselineskip}{10pt}
    \begin{algorithmic}[1]
    \While {$H$にあるs-t点素パスの最大本数が高々$K$}
        \State {$H$にある点素パスを計算して$k$を更新} \Comment {$O(km)$}
        \State {アルゴリズム2の計算:出力を$U$とする} \Comment {$O(km\log n)$}
        \State {アルゴリズム3の計算:出力を$W$とする} \Comment {$O(km\log n)$}
        \If {$U \neq (s,s,...,s)$}
            \State {$M_s := V_s(U) \cup U$}
            \State {$S := U$}
        \Else 
            \State {$M_s := \{s\}$}
        \EndIf

        \If {$W \neq (t,t,...,t)$}
            \State {$M_t := V_t(W) \cup W$}
            \State {$T := W$}
        \Else 
            \State {$M_t := \{t\}$}
        \EndIf

        \If {$M_s \cap M_t \neq \emptyset $または$M_s,M_t$間に辺がある}
            \State {\textbf{break}}
        \Else 
            \State {$M_s$を縮約して$s$にする}
            \State {$M_t$を縮約して$t$にする}
            \State {並行辺を一つの辺に置き換える}
            \State {$g(s) := g(s) + |M_s| - 1$} \Comment {$s$に縮約した頂点の数}
            \State {$g(t) := g(t) + |M_t| - 1$} \Comment {$t$に縮約した頂点の数}
        \EndIf
    \EndWhile \Comment {最大$K$回反復}
    \State {\textbf{return}} $(S,T)$
    \end{algorithmic}
    }
\end{algorithm}

アルゴリズムは$O(K^2m\log n)$で実行される

\begin{lemma}
    
\end{lemma}

\section*{分散案}
s-t点素パス計算:$\tilde{O}(K^{O(1)}(\sqrt{n} + D))$ラウンド \par
アルゴリズム2,3:$\tilde{O}(K^2n)$ラウンド \par
$M_s,M_t$の計算:2回のSAラウンドで出来る(できそう)\par
縮約した頂点は連結なので通信可能

\end{document}
\documentclass{jarticle}

%%% jdummy.def
%
\DeclareRelationFont{JY1}{mc}{it}{}{OT1}{cmr}{it}{}
\DeclareRelationFont{JT1}{mc}{it}{}{OT1}{cmr}{it}{}
\DeclareFontShape{JY1}{mc}{m}{it}{<5> <6> <7> <8> <9> <10> sgen*min
    <10.95><12><14.4><17.28><20.74><24.88> min10
    <-> min10}{}
\DeclareFontShape{JT1}{mc}{m}{it}{<5> <6> <7> <8> <9> <10> sgen*tmin
    <10.95><12><14.4><17.28><20.74><24.88> tmin10
    <-> tmin10}{}
\DeclareRelationFont{JY1}{mc}{sl}{}{OT1}{cmr}{sl}{}
\DeclareRelationFont{JT1}{mc}{sl}{}{OT1}{cmr}{sl}{}
\DeclareFontShape{JY1}{mc}{m}{sl}{<5> <6> <7> <8> <9> <10> sgen*min
    <10.95><12><14.4><17.28><20.74><24.88> min10
    <-> min10}{}
\DeclareFontShape{JT1}{mc}{m}{sl}{<5> <6> <7> <8> <9> <10> sgen*tmin
    <10.95><12><14.4><17.28><20.74><24.88> tmin10
    <-> tmin10}{}
\DeclareRelationFont{JY1}{mc}{sc}{}{OT1}{cmr}{sc}{}
\DeclareRelationFont{JT1}{mc}{sc}{}{OT1}{cmr}{sc}{}
\DeclareFontShape{JY1}{mc}{m}{sc}{<5> <6> <7> <8> <9> <10> sgen*min
    <10.95><12><14.4><17.28><20.74><24.88> min10
    <-> min10}{}
\DeclareFontShape{JT1}{mc}{m}{sc}{<5> <6> <7> <8> <9> <10> sgen*tmin
    <10.95><12><14.4><17.28><20.74><24.88> tmin10
    <-> tmin10}{}
\DeclareRelationFont{JY1}{gt}{it}{}{OT1}{cmbx}{it}{}
\DeclareRelationFont{JT1}{gt}{it}{}{OT1}{cmbx}{it}{}
\DeclareFontShape{JY1}{mc}{bx}{it}{<5> <6> <7> <8> <9> <10> sgen*goth
    <10.95><12><14.4><17.28><20.74><24.88> goth10
    <-> goth10}{}
\DeclareFontShape{JT1}{mc}{bx}{it}{<5> <6> <7> <8> <9> <10> sgen*tgoth
    <10.95><12><14.4><17.28><20.74><24.88> tgoth10
    <-> tgoth10}{}
\DeclareRelationFont{JY1}{gt}{sl}{}{OT1}{cmbx}{sl}{}
\DeclareRelationFont{JT1}{gt}{sl}{}{OT1}{cmbx}{sl}{}
\DeclareFontShape{JY1}{mc}{bx}{sl}{<5> <6> <7> <8> <9> <10> sgen*goth
    <10.95><12><14.4><17.28><20.74><24.88> goth10
    <-> goth10}{}
\DeclareFontShape{JT1}{mc}{bx}{sl}{<5> <6> <7> <8> <9> <10> sgen*tgoth
    <10.95><12><14.4><17.28><20.74><24.88> tgoth10
    <-> tgoth10}{}
\DeclareRelationFont{JY1}{gt}{sc}{}{OT1}{cmbx}{sc}{}
\DeclareRelationFont{JT1}{gt}{sc}{}{OT1}{cmbx}{sc}{}
\DeclareFontShape{JY1}{mc}{bx}{sc}{<5> <6> <7> <8> <9> <10> sgen*goth
    <10.95><12><14.4><17.28><20.74><24.88> goth10
    <-> goth10}{}
\DeclareFontShape{JT1}{mc}{bx}{sc}{<5> <6> <7> <8> <9> <10> sgen*tgoth
    <10.95><12><14.4><17.28><20.74><24.88> tgoth10
    <-> tgoth10}{}
\DeclareRelationFont{JY1}{gt}{it}{}{OT1}{cmr}{it}{}
\DeclareRelationFont{JT1}{gt}{it}{}{OT1}{cmr}{it}{}
\DeclareFontShape{JY1}{gt}{m}{it}{<5> <6> <7> <8> <9> <10> sgen*goth
    <10.95><12><14.4><17.28><20.74><24.88> goth10
    <-> goth10}{}
\DeclareFontShape{JT1}{gt}{m}{it}{<5> <6> <7> <8> <9> <10> sgen*tgoth
    <10.95><12><14.4><17.28><20.74><24.88> tgoth10
    <-> tgoth10}{}
\endinput
%%%% end of jdummy.def



\usepackage[dvipdfmx]{graphicx}
\usepackage{amsmath}			% math系
\usepackage{amssymb}			% math系
%\usepackage{float}				% 図表の挿入箇所を固定する[H]指定
\usepackage{cite}				% 参考文献
%\usepackage{url}				% 参考文献中のURL表記
\usepackage{algorithm}			% アルゴリズム環境
\usepackage{algorithmic}		% アルゴリズム環境
\usepackage{comment}			% コメントアウト環境
\usepackage{bm}	%太字形式のベクトル


\headsep=1.4cm  %本文上にスペースを空けたい場合は 20mm にする

% 定理環境
\usepackage{amsthm}
\theoremstyle{definition}
\newtheorem{theorem}{定理}
\newtheorem{lemma}{補題}
\newtheorem{definition}{定義}
\newtheorem{fact}{事実}
\newtheorem{corollary}{系}


% 先生コメント用
\usepackage[normalem]{ulem}
\usepackage{color}
\newcommand{\Izumi}[1]{\textcolor{blue}{#1}}
\newcommand{\Izurep}[2]{\textcolor{red}{\sout{#1}}{\Izumi{#2}}}

\begin{document}
    
\section*{準備}

\begin{definition}[Subgraph Aggregation]
    $G=(V,E)$をネットワークグラフとし、$\mathcal{P} =(P_1,...P_{|\mathcal{P}|})$をpartの集合とし、各$P_i$について$H_i$を$P_i$のノード上のGの連結部分グラフとする。必ずしもグラフ$G[P_i]$から誘導されるとは限らない。各部分グラフ$H_i$について、$V(H_i)$内のすべてのノードが部分グラフ$H_i$内の隣接ノードを認識し、それ以外は何も知らないと仮定する。すべてのノード$v\in \bigcup_iP_i$がO(log n)ビットの整数$x_v$を持ち、$\oplus$を長さO(log n)の整数に作用する結合関数とする。$P_i$内の各ノードは値$\bigoplus_{v\in P_i}x_v$、すなわち$P_i$内のすべての値$x_v$の集合$\oplus$を知りたいとする。このようなタスクをオペレーター$\oplus$におけるSubgraph Aggregationと呼ぶ。
\end{definition}

\begin{lemma}
    木幅が高々kのグラフ$G=(V,E)$と二つの頂点$s,t\subseteq V$を与えると、$k$点素s-tパスを見つけるか、サイズ$k$以下のs-tノードカットを$\tilde{O}(k^{O(1)}D)$ラウンドで出力することができる。前者の場合、すべてのノードは、それがパス上にあるかどうかを知っており、そうであれば、そのパス上のその前方と後方を知る。後者の場合、$k$点素パスが存在しないという事実と、ノードカットに含まれるかどうかをすべてのノードが知っている。
\end{lemma}

\begin{corollary}
    結合演算子$\oplus$について、$Q_G$がグラフ$G$とその直径$D$に依存するパラメータである場合、$\tilde{O}(Q_G)$ラウンドでSubgraph Aggregation問題を解くことができる。
    \item 全てのグラフ$G:Q_G = O(\sqrt{n}+D)$
    \item 種数$g$のグラフ$G:Q_G=O(\sqrt{g+1}D)$
    \item 木幅$k$のグラフ$G:Q_G=\tilde{O}(kD)$
    \item $H$をマイナーとして含まないグラフ$G:Q_G=\tilde{O}(f(H)\cdot D^2)$,$f$は$H$にのみ依存する関数
\end{corollary}

\begin{corollary}
    補題1のグラフ$G$を一般のグラフに置き換えると、最大$\ell$本のs-tパスは$\tilde{O}(\ell^{O(1)}(\sqrt{n}+D))$ラウンドで見つけることができる。
\end{corollary}

\newpage
\begin{lemma}[全域木]
    ネットワークグラフの連結部分グラフ$H\subseteq G$が与えられると、Gの全域木を$O(\log n)$SAラウンドで計算することができる。どのノードもスパニングツリーでの隣接ノードを認識している。
\end{lemma}

\begin{lemma}[根付き木収集]
    $G$の木$T$を考える。根$v_r \in V(T)$が与えられると、$O(\log n)$SAラウンドで$v_r$を根とする木$T$を計算することができ、$V(T)-v_r$の各ノードは$v_r$を根とする木Tの親を知る。さらに、各ノード$v_i$が整数$x_i$と共通結合演算子$\oplus$を知っていれば、各ノード$v_i$に部分木収集$\bigoplus_{j\in T(v_i)}x_j$を学習させることができる。ここで、$T(v_i)$は$v_i$をルートとする部分木、すなわち、根への経路に$v_i$を含む$T$内のすべてのノードである。
\end{lemma}

\begin{lemma}[経路収集]
    $G$の有向パス$P = \{v_1,...,v_\ell\}$を考える。ここで、各ノード$v_i$はそのパスの先頭ノードと後尾ノードを知っている。$O(\log n)$SAラウンドにおいて、各ノード$v_i$はiの値と経路内のそのインデックスを知ることができる。さらに、各ノード$v_i$が整数$x_i$と共通の結合演算子$\oplus$を知っていれば、各ノード$v_i$に先頭集約$\bigoplus_{j\leq i}xj$と後尾集約$\bigoplus_{j\geq i}x_j$を学習させることができる。
\end{lemma}

\begin{lemma}[s-tパス]
    連結部分グラフ$H\subseteq G$と2つの頂点$s,t\in V$が与えられると、$O(\log n)$SAラウンドで$G$における有向s-tパスを計算することができる。すべてのノードは、自分がパス上にあるかどうか、またある場合はそのパス上の先頭ノードと後尾ノードを認識している。
\end{lemma}
\newpage
\begin{definition}
    Gをグラフとし、$s,t\in V$とする。 ${f_1,...,f_k}$を$G$における一対の点素s−tパスの集合とする。
    この時、すべての$1\leq i\leq k$に対して、$w_i \in f_i$、$s \neq wi \neq t$であれば、$(f_1,f_k)$に関してタプル$(w_1,...,w_i)$をスライスと呼ぶ。
    XをVの任意の部分集合とし、$s,t = \in X$とする。$G[V\backslash X]$にs-tパスがない場合、$X$は$s$と$t$を分離すると言う。$s$と$t$を切り離すスライスをカットと呼ぶ。
\end{definition}

\begin{definition}
    ${f_1,...,f_k}$をs-t点素パスの集合とする。
    $U=(u_1,...u_k)$および$W=(w_1,...,w_k)$を$(f_1,...,f_k)$に関するスライスとし、すべての$1\leq i\leq k$について、$u_i$は$f_i$における$w_i$の前者または$u_i=w_i$とする。この時$U$は$W$より$S$に近いと言い、$U\preceq W$と書く。もし$1 \leq i \leq k$に対して、さらに$u_i \neq w_i$ならば、$U$は$W$より厳密に$s$に近いと言い、$U \prec W$と書く。同様に、$W$は$U$より厳密に$t$に近いと言う。便宜上、タプル$(s,s,...,s)$と$(t,t,...,t)$についても同様に上記を定義。したがって、例えば、$(s,s,...s)$はどのスライスよりもsに近いと言える。$"\preceq"$は一般的な順序の合計を定義するものではない。
\end{definition}

\begin{definition}
    $U$を任意のカットとする。$s$を含む$G [V\backslash U]$の連結成分の頂点セットとして$V_s(U)$を定義し、$t$を含む$G [V\backslash U]$の連結成分の頂点セットとして$V_t(U)$を定義し、$V_r$を$G [V \backslash U]$の残りの連結成分の頂点集合の和集合として定義する。すなわち、$s$も$t$も含まないものである(したがって、$V_r(u)$は空であり得る)。
\end{definition}

\begin{definition}
    $(f_1,...,f_k)$に関してUをスライスとする。
    $U\preceq X$であり、かつ$U\preceq X'\prec X$を満たすカット$X'$が存在しないようなカットを$X$とする。この時$U^+:=X$を定義する。\\
    上記のような$X$が存在しない場合は、$U^+:=(t,t,...,t)$とする。\\
    同様に、$Y\preceq U$かつ$U\preceq Y'\prec Y$を満たすカット$Y'$が存在しないようなカットを$Y$とする。この時、$U^-:=Y$を定義する。\\
    上記のような$Y$が存在しない場合は$U^-:=(s,s,...,s)$とする 
\end{definition}

\begin{lemma}
    $g(s),g(t)$を正の整数とする。$1\leq i\leq k$かつ$f_i$上で$u_i$の真後ろにある$u^+$について$U=(u_1,...,u_k)$と$W=(u_1,...,u_{i-1},u^+,u_{i+1},...,u_k)$はスライスであるとするこの時、\\
    $U^+=(t,t...,t)$または$|V_s(U^+)|+g(s)>|V_r(U^+)\cup V_t(U^+)+g(t)|$ならば$W^+=(t,t,....,t)$または
    $|V_s(W^+)|+g(s)>|V_r(W^+)|\cup V_t(W^+)|$
\end{lemma}

\begin{lemma}
    $U \preceq W$のような$U,W$を$(f_1,...,f_k)$に関するスライスとする。この時、$U^+ \preceq W^+$かつ$U^- \preceq w^-$である。
\end{lemma}

\begin{definition}
    $(A^*,S^*,B^*)$を頂点セパレータとする。この時、反復で選択された頂点$s,t$に対して、$s,t\in A$または$s,t\in B$である場合、アルゴリズム5のwhileループの反復は$(A^*,S^*,B^*)$に関して失敗する。そうでなければ、$(A^*,S^*,B^*)$に関して反復は成功したという。
\end{definition}
   

\end{document}
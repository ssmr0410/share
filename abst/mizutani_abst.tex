\documentclass[a4j,twoside]{jarticle}
\bibliographystyle{junsrt}

\input{jdummy.def} 
%\usepackage{graphicx}
\usepackage[dvipdfmx]{graphicx}
\usepackage{amsmath,amsthm}		% math系
\usepackage{amssymb}			% math系
%\usepackage{float}				% 図表の挿入箇所を固定する[H]指定
%\usepackage{cite}				% 参考文献
%\usepackage{url}				% 参考文献中のURL表記
\usepackage{algorithm}			% アルゴリズム環境
\usepackage{algorithmic}		% アルゴリズム環境
\usepackage{comment}			% コメントアウト環境
\usepackage{bm}	%太字形式のベクトル

% 定理環境
\newtheorem{theorem}{定理}
\newtheorem{lemma}{補題}
\newtheorem{definition}{定義}


% Unix上でのコンパイルはthesis_abst-jisを利用してください.
\usepackage{thesis_abst}

% マージンはプリンタによって変更
\addtolength{\oddsidemargin}{0mm}
\addtolength{\evensidemargin}{0mm}

% baselinestretchを変更すると上部枠の大きさが変わるのでおすすめしない
\renewcommand{\baselinestretch}{1}

\種別{情 報 工 学 科 卒 業}  % この行を消してはいけない
\学籍番号{26115142}
\氏名{水谷 龍誠}
%\英語氏名{} %未使用
\研究室{泉}
\系{ネットワーク} % 学生が所属する系を記入.教員(研究室)の所属する系ではない
\題目{平面的グラフに対する直径を計算する分散アルゴリズム} % 途中で改行 "\\" を挿入可
\年度{30} % !=年 発表は2月です
\begin{document}              % この行を消してはいけない
\twocolumn[\vspace*{9mm}]     % この行を消してはいけない
\begin{論文概要}              % この行を消してはいけない
\setcounter{page}{2}          % 表(左綴じ)は1、裏(右綴じ)は2を指定
%%%%%%% ここからアブスト本体 %%%%%%%

\section{はじめに}
ある$n$頂点のグラフ$G=(V,E)$が与えられたとき,頂点の部分集合$S \subset V$が,それを取り除くとグラフが非連結な二つ以上の部分グラフに分けられるとき,その頂点集合$S$は$G$の分離集合と呼ぶ.特に,$S$を取り除いた後におけるグラフの各連結成分がいずれも高々$\alpha n$個の頂点しか含まないとき$S$をグラフ$G$の$\alpha$-平衡分離集合と呼ぶ.$\alpha$が定数($\alpha = \Theta(1)$)であるようなサイズの小さい分離集合を発見できるとき,元のグラフ$G$に対する何らかの問題を,分離後の$\alpha n$頂点サイズのグラフにおける問題に分割統治法を用いて帰着できる場合がしばしば存在する.一般のグラフに対して,最小サイズの$\alpha$-平衡分離集合を求める問題はNP困難であることが知られているが~\cite{bui1992finding},いくつかの近似アルゴリズムの存在が知られている.本研究では,特に分散システム上の平衡分離集合発見問題を考える.すなわち,ネットワークのトポロジを問題の入力とみなし,その上での小さい平衡分離集合を発見するアルゴリズムを考える.分散システムのモデルとしては,単位時間当たりに1通信リンクあたりに伝送可能な情報量を$O(\log n)$ビットに制限した\textit{CONGEST}モデルを考える.本研究での提案アルゴリズムの基本アイデアは,BrandtとWattenhoferらによる,一般のグラフに対する平衡分離集合計算のための近似アルゴリズム\cite{brandt2017approximating}を分散システム上に実現することである.同アルゴリズムはサイズ$K$の$\alpha$-平衡分離集合を持つような入力インスタンスに対して,サイズ$O(\varepsilon^{-1}K^2\log^{1+o(1)}n)$の $(\alpha + \varepsilon)$-平衡分離集合を計算する.提案アルゴリズムは,このアルゴリズムと同等の近似性能を持つ解を$\tilde{O}(\varepsilon^{-1}K^3(K^{O(1)}+\ell)(\sqrt{n}+D))$ラウンドで出力する.

\section{諸定義}

\section{提案アルゴリズム}

\section{まとめ}


%%%%%% 以下の行は消さないこと %%%%%%%

\clearpage                       % この行を消すと最終ページの枠線消滅の危機
\end{論文概要}                   % この行を消してはいけない
\end{document}                   % この行を消してはいけない
